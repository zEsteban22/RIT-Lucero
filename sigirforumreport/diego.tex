%!TEX root = lucene4IR2016workshop_report.tex

\subsection*{Learning to Rank with Solr}
{\bf Diego Ceccarelli, Bloomberg}
On day two of the workshop, Diego started his talk by explaining that tuning
the relevance of a search system is often performed by ``experts'' who hand tune and craft
the weightings used for the different retrieval features.
However, this approach is manual, expensive to maintain, and based on intuitive or deep domain knowledge,
 rather than data. His working goal behind this project was to automate the process.
 He motivated the use of Learning To Rank, a technique that enables
 the automatic tuning of an information retrieval system.  He pointed out that sophisticated learned models can make
 more nuanced ranking decisions than a traditional ranking function when tuned
 in such a manner. This is the reason why, at Bloomberg, they have integrated a Learning to Rank
 component directly into Solr and contributed the code back\footnote{\url{https://issues.apache.org/jira/browse/SOLR-8542}}
 enabling others to easily build their own Learning To Rank systems.
During his talk, Diego presented the key concepts of
 Learning to Rank, how to evaluate the quality of the search in a production service,
 and finally described how the Solr Learning to Rank component works.

